% REQUIREMENTS.
% This document requires the following packages in order to process/compile it.
%
% Lance A. Endres's LaTeX library (LeLaTeX) available at the following location:
% https://github.com/lendres/LaTeX
%
% Perl
% https://strawberryperl.com/
% \documentclass{cnpcreport}
\documentclass{osdcreport}

\input{"Setup"}
\input{"Equations"}

\begin{document}
\input{"Title Page"}

\frontmatter{}
% TABLES.
% Generate the tables of things.
\tableofcontents{}
\listoffigures{}
\listoftables{}

% CONTENT.
\input{"Acknowledgements"}
\input{"Executive Summary"}

\mainmatter{}
% Glossary entries need to be defined before they can be referenced.
\input{"Glossary"}
\input{"Acronyms"}

% Writing notes.  Surround by \markupnote so it can be automatically removed.
\markupnote{\input{"Notes"}}

\input{"Introduction"}
\input{"Model Review"}
\input{"Recommended Procedures for Developing Test Cases"}
\input{"Test Cases"}
\input{"Aarsnes-Shor Model"}
\input{"Dixit Model"}
\input{"Results and Comparisons"}
\input{"Final Remarks and the Path Forward"}
\appendix{}
\input{"RGD Appendix"}
%\input{"Aarsnes-Shor Code Alterations"}
%\input{"Dixit Code Alterations"}
\input{"Additional Tests"}

% EXAMPLES
% Comment out for final versions.
%\input{"LaTeX Examples"}

\backmatter{}
% GLOSSARIES.
% The argument "nogroupskip" prevents adding extra space between the words that start with the same letter and the proceeding and following
% "set" (words with that star with the same letter).  I.e., there is more space between the "a"s and "b"s than is between the "a" entries.
% The "nopostdot" option removes the final dot/period that is otherwise automatically added.

% Standard glossary entries.
\printglossary[nogroupskip, nopostdot, type=main, title=Glossary]{}
\renewcommand*{\chaptertitle}{Glossary}

% Acronyms.
\printglossary[nogroupskip, nopostdot, type=\acronymtype]{}
\renewcommand*{\chaptertitle}{Acronyms}

% Use all glossary entries without specifically referencing them.
% Putting it before "\printgossaries" seems to cause a blank page to be injected.
\glsaddallunused{}

% PRINT A LIST OF NOTATIONS / NOMENCLATURE.
% Argument is a text/tex file containing lines in the form of:
% \addnotation{Variable}{Meaning of variable.}
%\input{"List of Notations"}

% BIBLIOGRAPHIES.
% These are found in the LeLaTeX package (see above).
\bibliography{strings, le}

\end{document} 