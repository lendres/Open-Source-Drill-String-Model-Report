\chapter{Test Cases}
\label{ch:testcases}
Test Cases were developed in this study for the purpose of comparing and validating different drill string models.  The Test Cases are meant to be general and provide a reliable means of cross-checking and comparing the performances of various models, not just those specific to the current study.  As such, the key parameters, borehole design, and other information required to reproduce the Test Cases is provided. For convenience, the tables in this chapter contains both imperial and metric units.

\section{Overview of Test Cases}
In total, six Test Cases are presented. Test Cases 1 and 3 encompass scenarios involving vertical wells, with the distinction being the presence of BHA components. Similarly, Test Cases 2 and 4 are deviated well scenarios with a 60\textdegree{} inclination angle. The difference between Test Cases 2 and 4 is the exitance of BHA components. In addition, to analyze the effect of frictional factors, Test Cases 2 and 4 are further divided into \emph{a} and \emph{b} subcases (2a, 2b, 4a, 4b). The friction factors are tested in the deviated wells because in a vertical well gravity does not create contact force (normal force). All the current Test Cases are for off-bottom scenarios. \tablename~\ref{Test_case_summary} summarizes the scenarios for each Test Case.
\begin{table}
  \centering
  \begin{tabular}{|c|c|c|c|c|c|c|}
    \hline
    \tablecolumnheadervlinesone{Test Case} & \tablecolumnheadervlinestwo{Well Type} & \tablecolumnheadervlinestwo{Static} & \tablecolumnheadervlinestwo{Dynamic} & \tablecolumnheadervlinestwo{Viscous} & \tablecolumnheadervlinestwo{BHA} & \tablecolumnheadervlinestwo{On} \\[-7pt]
                                           &                                        & \tablecolumnheadervlinestwo{FF}     & \tablecolumnheadervlinestwo{FF}      & \tablecolumnheadervlinestwo{Damping} & &     \tablecolumnheadervlinestwo{Bottom} \\
    \hline
    Test Case 1 & Vertical & 0 & 0 & 0 & N & N \\
    \hline
    Test Case 2a & Deviated (60\textdegree{}) & 0.5 & 0.5 & 0 & N & N \\
    \hline
    Test Case 2b & Deviated (60\textdegree{}) & 0.5 & 0.25 & 0 & N & N \\
    \hline
    Test Case 3 & Vertical & 0 & 0 & 0 & Y & N \\
    \hline
    Test Case 4a & Deviated (60\textdegree{}) & 0.5 & 0.5 & 0 & Y & N \\
    \hline
    Test Case 4b & Deviated (60\textdegree{}) & 0.5 & 0.25 & 0 & Y & N \\
    \hline
  \end{tabular}
  \caption[Scenarios of the defined Test Cases]{Scenarios of the defined Test Cases.}
  \label{Test_case_summary}
\end{table}
Additionally, the viscous damping is neglected for the simplicity of the tests in this project, but it can be included in future studies.  The tests were conducted by assuming the top drive velocity is increased from an initial 0 $RPM$ to 40 $RPM$ at 1 second. The velocity is maintained for the remainder of the simulation. The top drive velocity profile is shown in \figurename~\ref{figure_topdrive_VSP}
\begin{figure}
  \centering
  \includegraphics[width=3.75in]{TopdriveVSP}
  \caption[Top drive velocity profile for the Test Cases]{Top drive velocity profile for the Test Cases.}\label{figure_topdrive_VSP}
\end{figure}

\section{Terminology}
\subsection{With BHA vs Without BHA}
The Test Cases enable comparing drilling scenarios with and without the BHA components. This allows the influence of the BHA to be examined for each model and between models by comparing the two cases.

\begin{definition}{Without BHA}
Analysis with a single drill pipe configuration to maintain simplicity and establish a baseline. This configuration represents the most straightforward arrangement, with a solitary pipe size running from the surface to the drill bit. In this case, the weight and diameter of the entire drill pipe are uniform and tool joints are ignored.
\end{definition}
\begin{definition}{With BHA}
This case is based on the Without BHA case with the notable difference that a BHA is added.  The weight and diameter of the drill pipe is are uniform, however, BHA components with different sizes are included in the analysis.  Tool joints are ignored.
\end{definition}

%In the next subsection we will explore additional configurations with varying BHA designs and properties to further investigate their influence on drilling efficiency and performance. These test cases will enable us to optimize drilling operations and develop a comprehensive understanding of how the presence or absence of the BHA affects the system's behavior.

\subsection{Friction Factors}
Many drill string models take into account both static and dynamic friction between the drill string and wellbore and they play a crucial role in affecting the drilling performance and behavior.  The friction factors (FF) will affect the drill string motion in a deviated well because of the interaction between the drill string and the wellbore wall.  Therefore, the friction factors need to be assessed by the Test Cases.
%but the disparity in FF values is expected to be more pronounced in inclined wells compared to vertical ones.
%This results in differing frictional forces acting on the drill string as it traverses the wellbore. As a consequence, the FF values play a crucial role in affecting the drilling performance and behavior.

\begin{definition}{Zero FF}
Tests conducted with the friction factors set to zero.
\end{definition}
\begin{definition}{Same FF}
Tests conducted with the same value of static and dynamic friction factors.
\end{definition}
\begin{definition}{Different FF}
Tests conducted with different values of static and dynamic friction factors.
\end{definition}

\section{Test Case 1 - Vertical Well Without BHA - Zero FF}
The model parameters and schematic of the wellbore surveys and drill string components for Test Case 1 are shown in \tablename~\ref{table_verticalwell_input} and \figurename~\ref{figure_verticalwell}. The Test Case uses a vertical well with measured depth of 2000 $m$ and bit depth of 1800 $m$.

\begin{figure}
  \centering
  \includegraphics[width=1.5in]{VerticalWellConfig}
  \caption[Schematic of wellbore and drill string for Test Case 1]{Schematic of wellbore and drill string for Test Case 1.}\label{figure_verticalwell}
\end{figure}


\begin{table}
    \centering
	\begin{testcasetable}
		$\rho$ & 490.6 $lb/ft^3$ & 7850 $kg/m^3$ & Drill pipe density \\
		\hline
		$G_{dp}$ & 1.67$\cdot$10$^{9}$ $lbf/ft^2$ & 7.99$\cdot$10$^{10}$ $Pa$  & Shear modulus \\
		\hline
		$OD_{dp}$ & 5.88 $in$ & 0.15 $m$ & Drill pipe outer diameter \\
		\hline
		$ID_{dp}$ & 5.00 $in$ & 0.127 $m$ & Drill pipe inner diameter  \\
		\hline
		$BD$ & 5905.5 $ft$ & 1800 $m$ & Bit depth \\
		\hline
		$MD$ & 6561.7 $ft$ & 2000 $m$ & Measured depth \\
		\hline
	\end{testcasetable}
	\caption[Input parameters for Test Case 1]{Input parameters for Test Case 1, a vertical well without BHA components.}\label{table_verticalwell_input}
\end{table}

\section{Test Case 2 - Deviated Well Without BHA}
\subsection{Test Case 2a - Same FF Values}
This Test Case uses a deviated well with a simple configuration of the drill string. The measured depth of the well is 4000 $m$ with a 60$^{\circ}$ inclination from the kick off point of 1500 $m$. The drill bit is off-bottom and located at 2500 $m$. The schematic view of the wellbore and drill string are depicted in \figurename~\ref{figure_wellconfig_inclined}. The static and dynamic friction factors were both set to 0.5. The parameters for the test are summarized in \tablename~\ref{table_Inclinedwell_2a_input}.

\begin{figure}
  \centering
  \includegraphics[width=4in]{InclinedWellConfig}
  \caption[Schematic of Test Case 2]{Schematic of the wellbore (left) and drill string (right) for Test Case 2.}
  \label{figure_wellconfig_inclined}
\end{figure}

\begin{table}
    \centering
	\begin{testcasetable}
		$\rho$ & 490.6 $lb/ft^3$ & 7850 $kg/m^3$ & Drill pipe density \\
		\hline
		$G_{dp}$ & 1.67$\cdot$10$^{9}$ $lbf/ft^2$ & 7.99$\cdot$10$^{10}$ $Pa$  & Shear modulus \\
		\hline
		$OD_{dp}$ & 5.88 $in$ & 0.15 $m$ & Drill pipe outer diameter \\
		\hline
		$ID_{dp}$ & 5.00 $in$ & 0.127 $m$ & Drill pipe inner diameter  \\
		\hline
		$\mu_{s}$ & 0.5 & 0.5 & Static friction factor \\
		\hline
		$\mu_{d}$ & 0.5 & 0.5 & Dynamic friction factor \\
		\hline
		$w_c$ & 10 $RPM$ & 10 $RPM$ & Friction critical velocity \\
		\hline
		$\theta$ & 60$^{\circ}$ & 60$^{\circ}$ & Inclination\\
		\hline
		$KOP$ & 4921.3 $ft$ & 1500 $m$ & Kick off point \\
		\hline
		$EOB$ & 6889.8 $ft$ & 2100 $m$ & End of bend \\
		\hline
		$BD$ & 8202.1 $ft$ & 2500 $m$ & Bit depth \\
		\hline
		$MD$ & 13123.4 $ft$ & 4000 $m$ & Measured depth \\
		\hline
	\end{testcasetable}
	\caption[Input parameters for Test Case 2a]{Input parameters for Test Case 2a, a deviated well without BHA components and has the same dynamic and static friction factor values.}\label{table_Inclinedwell_2a_input}
\end{table}

\subsection{Test Case 2b - Different FF Values}
This Test Case uses the exact same configuration as Test Case 2a except the dynamic friction factor was reduced from 0.5 to 0.25.  \tablename~\ref{table_Inclinedwell_2b_input} summarizes the input parameters.

\begin{table}
	\centering
	\begin{testcasetable}
		$\rho$ & 490.6 $lb/ft^3$ & 7850 $kg/m^3$ & Drill pipe density \\
		\hline
		$G_{dp}$ & 1.67$\cdot$10$^{9}$ $lbf/ft^2$ & 7.99$\cdot$10$^{10}$ $Pa$  & Shear modulus \\
		\hline
		$OD_{dp}$ & 5.88 $in$ & 0.15 $m$ & Drill pipe outer diameter \\
		\hline
		$ID_{dp}$ & 5.00 $in$ & 0.127 $m$ & Drill pipe inner diameter  \\
		\hline
		$\mu_{s}$ & 0.5 & 0.5 & Static friction factor \\
		\hline
		$\mu_{d}$ & 0.25 & 0.25 & Dynamic friction factor \\
		\hline
		$w_c$ & 10 $RPM$ & 10 $RPM$ & Friction critical velocity \\
		\hline
		$\theta$ & 60$^{\circ}$ & 60$^{\circ}$ & Inclination \\
		\hline
		$KOP$ & 4921.3 $ft$ & 1500 $m$ & Kick off point \\
		\hline
		$EOB$ & 6889.8 $ft$ & 2100 $m$ & End of bend \\
		\hline
		$BD$ & 8202.1 $ft$ & 2500 $m$ & Bit depth \\
		\hline
		$MD$ & 13123.4 $ft$ & 4000 $m$ & Measured depth \\
		\hline
	\end{testcasetable}
	\caption[Input parameters for Test Case 2b]{Input parameters for Test Case 2b, a deviated well without BHA components and different dynamic and static friction factor values.}
	\label{table_Inclinedwell_2b_input}
\end{table}

\section{Test Case 3 - Vertical Well with BHA - Zero FF}
This Test Case includes BHA components added to the drill string. The BHA's extra components and increased components size result in an increase in both the weight and size of the drill string. The wellbores is the same as Test Case 1 and is shown in \figurename~\ref{Vert_well_conf_BHA}. The input parameters can be found in the \tablename~\ref{Input Parameters TC3}.

\begin{figure}
  \centering
  \includegraphics[width=2in]{VerticalWellConfigBHA}
  \caption[Schematic of well and drill string for Test Case 3]{Schematic of well and drill string for Test Case 3.}
  \label{Vert_well_conf_BHA}
\end{figure}


\begin{table}
    \centering
	\begin{testcasetable}
		$\rho$ & 490.6 $lb/ft^3$ & 7850 $kg/m^3$ & Drill pipe density \\
		\hline
		$G_{dp}$ & 1.67$\cdot$10$^{9}$ $lbf/ft^2$ & 7.99$\cdot$10$^{10}$ $Pa$  & Shear modulus \\
		\hline
		$OD_{dp}$ & 5.88 $in$ & 0.15 $m$ & Drill pipe outer diameter \\
		\hline
		$ID_{dp}$ & 5.00 $in$ & 0.127 $m$ & Drill pipe inner diameter  \\
		\hline
	    $OD_{HWDP}$ & 4.50 $in$ & 0.1143 $m$ & Heavy weight drill pipe outer diameter \\
	    \hline
	    $ID_{HWDP}$ & 2.50 $in$ & 0.0635 $m$ & Heavy weight drill pipe inner diameter \\
	    \hline
	    $L_{HWDP}$ & 60 $ft$ & 18.30 $m$ & Length of heavy weight drill pipe \\
	    \hline
	    $OD_{DC}$ & 6.00 $in$ & 0.1524 $m$ & Drill collars outer diameter \\
	    \hline
	    $ID_{DC}$ & 2.00 $in$ & 0.0508 $m$ & Drill collars inner diameter \\
	    \hline
	    $L_{DC}$ & 270 $ft$ & 82.30 $m$ & Length of drill collars \\
	    \hline
		$BD$ & 5905.5 $ft$ & 1800 $m$ & Bit depth \\
		\hline
		$MD$ & 6561.7 $ft$ & 2000 $m$ & Measured depth \\
		\hline
    \end{testcasetable}
  \caption[Input parameters for Test Case 3]{Input parameters for Test Case 3, a vertical well with BHA components.}
  \label{Input Parameters TC3}
\end{table}

\section{Test Case 4 - Deviated Well With BHA}
\subsection{Test Case 4a - Same FF Values}
This Test Case uses the same input parameters from Test Case 2 but includes a BHA. The wellbores is the same as Test Case 2 and is shown in \figurename~\ref{figure_wellconfig_inclined_BHA}. The input parameters can be found in the \tablename~\ref{table_Inclinedwell_4a_input}.

\begin{figure}
  \centering
  \includegraphics[width=4in]{InclinedWellConfigBHA}
  \caption[Schematic view of Test Case 4]{Schematic view of wellbore and drill string for Test Case 4.}\label{figure_wellconfig_inclined_BHA}
\end{figure}

\subsection{Test Case 4b - Same FF Values}
This Test Case uses the same configuration as Test Case 4a except the dynamic friction factor was reduced from 0.5 to 0.25. \tablename~\ref{table_Inclinedwell_4b_input} summarizes the input parameters.

\begin{table}
	%\vspace{-1pt} % A trick to get these two floats on the same page, not recommended.
    \centering
	\begin{testcasetable}
		$\rho$ & 490.6 $lb/ft^3$ & 7850 $kg/m^3$ & Drill pipe density \\
		\hline
		$G_{dp}$ & 1.67$\cdot$10$^{9}$ $lbf/ft^2$ & 7.99$\cdot$10$^{10}$ $Pa$  & Shear modulus \\
		\hline
		$OD_{dp}$ & 5.88 $in$ & 0.15 $m$ & Drill pipe outer diameter \\
		\hline
		$ID_{dp}$ & 5.00 $in$ & 0.127 $m$ & Drill pipe inner diameter  \\
		\hline
		$OD_{HWDP}$ & 4.50 $in$ & 0.1143 $m$ & Heavy weight drill pipe outer diameter \\
		\hline
		$ID_{HWDP}$ & 2.50 $in$ & 0.0635 $m$ & Heavy weight drill pipe inner diameter \\
		\hline
		$L_{HWDP}$ & 60 $ft$ & 18.30 $m$ & Length of heavy weight drill pipe \\
		\hline
		$OD_{DC}$ & 6.00 $in$ & 0.1524 $m$ & Drill collars outer diameter \\
		\hline
		$ID_{DC}$ & 2.00 $in$ & 0.0508 $m$ & Drill collars inner diameter \\
		\hline
		$L_{DC}$ & 270 $ft$ & 82.30 $m$ & Length of drill collars \\
		\hline
		$\mu_{s}$ & 0.5 & 0.5 & Static friction factor \\
		\hline
		$\mu_{d}$ & 0.5 & 0.5 & Dynamic friction factor \\
		\hline
		$w_c$ & 10 $RPM$ & 10 $RPM$ & Friction critical velocity \\
		\hline
		$\theta$ & 60$^{\circ}$ & 60$^{\circ}$ & Inclination \\
		\hline
		$KOP$ & 4921.3 $ft$ & 1500 $m$ & Kick off point \\
		\hline
		$EOB$ & 6889.8 $ft$ & 2100 $m$ & End of bend \\
		\hline
		$BD$ & 8202.1 $ft$ & 2500 $m$ & Bit depth \\
		\hline
		$MD$ & 13123.4 $ft$ & 4000 $m$ & Measured depth \\
		\hline
   \end{testcasetable}
   \caption[Input parameters for Test Case 4a]{Input parameters for Test Case 4a, a deviated well with BHA components and with the same static and dynamic friction factors.}
   \label{table_Inclinedwell_4a_input}
\end{table}

\begin{table}
	\centering
	\begin{testcasetable}
		$\rho$ & 490.6 $lb/ft^3$ & 7850 $kg/m^3$ & Drill pipe density \\
		\hline
		$G_{dp}$ & 1.67$\cdot$10$^{9}$ $lbf/ft^2$ & 7.99$\cdot$10$^{10}$ $Pa$  & Shear modulus \\
		\hline
		$OD_{dp}$ & 5.88 $in$ & 0.15 $m$ & Drill pipe outer diameter \\
		\hline
		$ID_{dp}$ & 5.00 $in$ & 0.127 $m$ & Drill pipe inner diameter  \\
		\hline
		$OD_{HWDP}$ & 4.50 $in$ & 0.1143 $m$ & Heavy weight drill pipe outer diameter \\
		\hline
		$ID_{HWDP}$ & 2.50 $in$ & 0.0635 $m$ & Heavy weight drill pipe inner diameter \\
		\hline
		$L_{HWDP}$ & 60 $ft$ & 18.30 $m$ & Length of heavy weight drill pipe \\
		\hline
		$OD_{DC}$ & 6.00 $in$ & 0.1524 $m$ & Drill collars outer diameter \\
		\hline
		$ID_{DC}$ & 2.00 $in$ & 0.0508 $m$ & Drill collars inner diameter \\
		\hline
		$L_{DC}$ & 270 $ft$ & 82.30 $m$ & Length of drill collars \\
		\hline
		$\mu_{s}$ & 0.5 & 0.5 & Static friction factor \\
		\hline
		$\mu_{d}$ & 0.25 & 0.25 & Dynamic friction factor \\
		\hline
		$w_c$ & 10 $RPM$ & 10 $RPM$ & Friction critical velocity \\
		\hline
		$\theta$ & 60$^{\circ}$ & 60$^{\circ}$ & Inclination \\
		\hline
		$KOP$ & 4921.3 $ft$ & 1500 $m$ & Kick off point \\
		\hline
		$EOB$ & 6889.8 $ft$ & 2100 $m$ & End of bend \\
		\hline
		$BD$ & 8202.1 $ft$ & 2500 $m$ & Bit depth \\
		\hline
		$MD$ & 13123.4 $ft$ & 4000 $m$ & Measured depth \\
		\hline
	\end{testcasetable}
	\caption[Input parameters for Test Case 4b]{Input parameters for Test Case 4b, a deviated well with BHA components and with different static and dynamic friction factors.}
	\label{table_Inclinedwell_4b_input}
\end{table}
\pagebreak