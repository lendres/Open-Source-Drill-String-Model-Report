\section{Test Case}
Exxon mobil model and Aarsnes-shor model (MATLAB ver. and PYTHON ver.) are compared with simple well and drill string configurations. The model parameters and schematic of the wellbore surveys and drill string components are shown in Table \tablename~\ref{table_verticalwell_input} and \ref{figure_verticalwell}. The Exxon mobil model and MATLAB ver. A-S model uses metric units while PYTHON ver. A-S model uses imperial units. For the future convenience, the Table in this chapter contains both imperial and metric units.

\begin{figure}[!hbt]
  \centering
  \includegraphics[width=1.5in]{VerticalWellConfig}
  \caption[Schematic of well and drill string for model comparison.]{Schematic of well and drill string for model comparison.}\label{figure_verticalwell}
\end{figure}

\begin{table}[!hbt]
\centering
\begin{tabular}{|c|c|c|c|}
\hline
Parameter & value (imperial units) & value (metric units) & description\\                                                              
\hline
$\rho$ & $490.6\;lb/ft^3$ & $7850\;kg/m^2$ & Drill pipe density \\                                                  
\hline
G & $1.67e^9\;lbf/ft^3$ & $7.99e^{10}\;Pa$  & Shear modulus \\                                                  
\hline
OD & 5.88 in & 0.15 m & drill pipe outer diameter\\                                                       
\hline
ID & 5.00 in & l.127 m & drill pipe inner diameter  \\                                                      
\hline
MD & 6561 ft & 2000 m & measured depth\\                                                              
\hline
TVD & 6561 ft & 2000 m & total vertical depth\\
\hline
Bit depth & 5905 ft & 1800 m & - \\ 
\hline
\end{tabular}
\caption[Well survey data for model comparison (vertical well)]{Well survey and drill string data for model comparison (vertical well without BHA components)}\label{table_verticalwell_input}
\end{table}
The test was conducted by assuming the top drive velocity to be increased from 0 RPM to 40 RPM at 1 second and maintained the velocity for rest of the time. The top drive velocity is shown in \figurename~\ref{figure_topdrive_VSP}

\begin{figure}[!hbt]
  \centering
  \includegraphics[width=3in]{TopdriveVSP}
  \caption[Top drive set velocity]{Top drive set velocity}\label{figure_topdrive_VSP}
\end{figure}

Comparison between A-S model (both Matlab ver. and Python ver.), and Exxon mobil model was conducted with given input in \tablename~\ref{table_verticalwell_input}. The modified code of PYTHON ver. with removed top drive was used for the comparison. First, for the simplicity, the viscous damping was assumed to be zero. Moreover, the coulomb friction is also zero since there is no contact between drill string and borehole (vertical well). Therefore the source term of the governing equation will be zero in equation \equationname~\ref{AS-motion} and the parameters that affects the model will be the density, shear modulus of drill pipe.

The comparison between different models are shown in \figurename~\ref{figure_modelcomparison_vertical_torque}. As can be seen from the results all the model showed  oscillation of the top drive torque, however, significant difference were observed in A-S model PYTHON ver. while MATLAB ver., and exxon mobil model shwed maximum torque about 2000 lb-ft, torque of the PYTHON ver. reached 10,000 lb-ft. Also, the differences in the oscillation frequencies were observed. The frequency of oscillation were 0.32, 0.45, and 0.125 for MATLAB ver., and Exxon mobil model, and PYTHON ver., respectively. Although MATLAB ver. and Exxon mobil model showed similar response, the convergence time of this oscillation were significantly shorter in MATLAB ver. The comparison of the converging time between MATLAB ver. and PYTHON ver. is illustrated in \figurename~\ref{figure_modelcomparison_vertical_convergence}. The comparison of the model are summerized in Table....

\begin{figure}[hbt!]
  \centering
  \includegraphics[width=6in]{modelcomparison_vertical_torque}
  \caption[Comparison of torque of top drive and bit without viscous damping and BHA components in vertical well]{Comparison of torque of top drive and bit without viscous damping and BHA components in vertical well. the input parameteres for drill string is summarized in \tablename~\ref{table_verticalwell_input}. a): A-S model (MATLAB ver.), b): Exxon mobil model, c) A-S model (PYTHON ver.), and d): A-S model (PYTHON ver.) with smooth increase of top drive velocity.}\label{figure_modelcomparison_vertical_torque}
\end{figure}

\begin{figure}
  \centering
  \includegraphics[width=6.5in]{modelcomparison_vertical_convergence}
  \caption[Comparison of torsional vibration convergence time without viscous damping and BHA components in vertical well.]{Comparison of torsional vibration convergence time without viscous damping and BHA components in vertical well. a): A-S model MATLAB ver., b): A-S model PYTHON ver.}\label{figure_modelcomparison_vertical_convergence}
\end{figure}

%\begin{table}[!hbt]
%\centering
%\begin{tabular}{|c|c}|c|c|}
%\hline
% & Exxon mobil model & A-S model (MATLAB ver.) & A-S model (PYTHON ver.) \\                                                              
%\hline
%Drill string vibration convergence & - & $<$ 150s & $>$ 30min \\                                                  
%\hline
%Vibration frequency & 0.32 & 0.45  & 0.125 \\                                                  
%\hline
%Maximum torque on top drive (lb-ft) & 2000 in & 2000 & \\                                                       
%\hline
%\end{tabular}
%\caption[Summary of model comparison without damping and BHA components in vertical well.]{Summary of model comparison without damping and BHA components in vertical well.}\label{table_modelcomparison_vertical_input}
%\end{table}

\newpage
\subsection{Case 2 - Inclined well}

The model was tested with inclined well with simple configuration of drill string. The MD of the well is 4000 m with 60 deg inclination. The drill bit is off-bottom where located at 2500 m depth. The Schematic view of wellbore and drill string are depicted in \figurename~\ref{figure_wellconfig_inclined}. Also, the top drive velocity is increased to 40 RPM at 1 second which is same with test case 1 (\figurename\ref{figure_topdrive_VSP}). For test case 2, the viscous damping is neglected for the simplicity of the test. However, Coulomb friction is considered to investigate the stick-slip occurrence during drilling. The parameters for the test are summarized in \tablename~\ref{table_Inclinedwell_input}.

\begin{figure}[!hbt]
  \centering
  \includegraphics[width=4in]{InclinedWellConfig}
  \caption[Schematic view of test case 2.]{Schematic view of wellbore and drill string for test case 2.}\label{figure_wellconfig_inclined}
\end{figure}

 \begin{table}[!hbt]
\centering
\begin{tabular}{|c|c|c|c|}
\hline
Parameter & Value (imperial units) & Value (metric units) & description\\                                                              
\hline
$OD_{dp}$ & 5.88 in & 0.15 m & drill pipe outer diameter\\                                                       
\hline
$ID_{dp}$ & 5.00 in & 0.l127 m & drill pipe inner diameter  \\                                                      
\hline
$\rho_{dp}$ & $490.6\;lb/ft^3$ & $7850\;kg/m^2$ & Drill pipe density \\                                                  
\hline
$G_{dp}$ & $1.27e^{9}\;lb/ft^3$ & $6.10e^{10}\;pa$ & drill pipe shear modulus\\                                                              
\hline
$\mu$ & 0.5 & 0.5 & Static friction factor\\
\hline
$fRat$ & 0.5 & 0.5 & Static friction/Dynamic friction\\
\hline
$w_c$ & 10 RPM & 10 RPM & Cut-off angular velocity\\
\hline
\end{tabular}
\caption[Input parameters for test case 2.]{Input parameters for test case 2. fRat only included in MATLAB ver. model, have to figure how to convert it to dynamic friction factor***).}\label{table_Inclinedwell_input}
\end{table}

The results from A-S models are shown in \figurename~\ref{figure_testcase2_ASmodel}. Both model captured the stick-slip event during of-bottom drill string rotation. However, similar to test case 1 (have to rename it later: vertical well test), the top drive torque amplitude and the frequency of the oscillation showed significant differences.
 
\begin{figure}[!hbt]
  \centering
  \includegraphics[width=6.5in]{Testcase2_ASmodel}
  \caption[Result from A-S model. (test case 2)]{Simulation results of A-S model for test case 2, a): MATLAB ver, and b): PYTHON ver. Stick-slip was observed from the model.}\label{figure_testcase2_ASmodel}
\end{figure}


A-S model
\begin{numberedlist}
	\item Difference in torque amplitude
	\item Difference in oscillation frequencies
	\item Bit model - constant torque
	\item Comparison using color map?
\end{numberedlist}

\newpage
\subsection{Case 3 - Vertical Well with BHA}

\begin{figure}[!hbt]
  \centering
  \includegraphics[width=2in]{VerticalWellConfigBHA}
  \caption{Schematic of well and drill string for model comparison}\label{Vert_well_conf_BHA}
\end{figure}


\begin{table}
  \centering
  \begin{tabular}{|c|c|c|p{2.2in}|}
    \hline
    % after \\: \hline or \cline{col1-col2} \cline{col3-col4} ...
    Parameter & Value (imperial units) & Value (metric units) & Description \\
    \hline
    $OD_{HWDP}$ & 4.50 in & 0.1143 m & Heavy weight drill pipe outer diameter \\
    \hline
    $ID_{HWDP}$ & 2.50 in & 0.0635 m & Heavy weight drill pipe inner diameter \\
    \hline
    $OD_{DC}$ & 6.00 in & 0.1524 m & Drill collars outer diameter \\
    \hline
    $ID_{DC}$ & 2.00 in & 0.0508 m & Drill collars inner diameter \\
    \hline
    $\rho_{dp}$ & 490.6 $lb/ft^{3}$ & 7850 $kg/m^{3}$ & Drill pipe density \\
    \hline
    $G_{dp}$ & 1.27$e^{9}\;lb/ft^{2}$ & 6.08$e^{10} pa$ & Drill pipe shear modulus \\
    \hline
  \end{tabular}
  \caption{Input parameters of BHA for test case 3}\label{Input Parameters TC3}
\end{table} 

In this specific scenario, an additional configuration of the drill string included a Bottom Hole Assembly (BHA). The BHA introduced extra components, resulting in an increase in both the weight and size of the drill string. However, the well survey data for the vertical well remained the same, except for the incorporation of the BHA. The well design, depicted in \figurename~\ref{Vert_well_conf_BHA}, remained unchanged, while the parameters for the top drive were kept constant. For the sake of simplicity, the influence of viscous damping was disregarded. The specific input parameters can be found in the provided \tablename~\ref{Input Parameters TC3}.

After running the simulation we get the following results. Top-drive torque remains constant while Rotational speed of the drill bit plays around
