\chapter{Dixit Code Alterations}
\label{ch:appendixexxonmobil}
This appendix contains a list of modifications made to the Dixit model during the testing and evaluation.

\section{1st Cell of \codefont{scratch\_pad.ipynb}}

The code was organized by creating separate folders for the input and output files, facilitating easier usage and data storage. Additional modifications were made to the original source code to create compatibility with different operating systems, as it was previously designed exclusively for Windows and was ``Path'' sensitive.

\begin{codemodifications}

\begin{codemodification}{1}
Importing \textcode{os} library for further use.
\end{codemodification}

\begin{codemodification}{5, 11}
The lines which are from original file were commented out and replaced with new code lines.
\end{codemodification}

\begin{codemodification}{7-8, 11}
In the original files, the Path/Location of the files involved the user name and would need to be renamed for every user.  New variables were added that will locate the current folder, the output and input files along with their directories.
\end{codemodification}

\end{codemodifications}

\section{2nd Cell of \codefont{scratch\_pad.ipynb}}
If you have the output file from previously run simulations, you can skip the first cell and directly run this second cell without rerunning the simulation. The modifications were primarily made in location part.
\begin{codemodifications}

\begin{codemodification}{1}
This line was removed since it was duplicate.
\end{codemodification}

\begin{codemodification}{5}
Importing \textcode{os} library for further use.
\end{codemodification}

\begin{codemodification}{6-9}
The original line was replaced with new lines which allow to importing existing data.  The original line of code was ``Path'' sensitive.
\end{codemodification}

\end{codemodifications}

\section{File \codefont{bitrock.py}}
During testing, we encountered an issue with the ``on-bottom'' case related to the bit-rock interaction model. In the ``drilling'' mode a constant value of depth of cut (DOC) and bit depth was observed. The cause was identified as an improper assignment of bit depth to bit displacement. A slight modification was made to incorporate the bit depth into the bit displacement calculation.

%This change, along with other adjustments, successfully activated the ``drilling'' mode, enabling dynamic changes in both hole depth and non-zero DOC.

\begin{codemodifications}

\begin{codemodification}{29}
Originally, the variable \textcode{bit\_depth} was set to \textcode{z}. However, since \textcode{z} signifies displacement and not the actual bit depth, it does not accurately represent the true bit depth. To obtain the correct bit depth, the displacement was needed to be added to the original bit depth, resulting in a list of updated bit depths for each calculation time step.
\end{codemodification}
\end{codemodifications}

A full comparison of outputs between original and modified code is shown in \figurename~\ref{Comaprison_on_bottom}.  The outputs from the original are in the left column and modified version are in the right column. For this test, the bit is initially off-bottom and is transitioned to the on-bottom case.  The drilling process commences at approximately 30 seconds. When the bit makes contact with the bottom, the bit torque, top-drive torque, WOB, and DOC all increase. Furthermore, the hole depth starts to change simultaneously with the bit depth, indicating the initiation of the drilling mode.
\begin{figure}
	\centering
	\includegraphics[width=\linewidth]{comp_on_bottom}
    \caption[Output comparison of the original and modified model Dixit model]{Comparison of the outputs from the original and modified versions of the Dixit model. Plots in left and right column are the results from the original and modified codes, respectively.}
	\label{Comaprison_on_bottom}
\end{figure} 