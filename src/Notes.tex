\chapter{Notes}
\warning{Remove before final printing.}
\section{To Do}
\begin{bulletedlist}
	\item Make note of the reflections at the bottom of the angular velocities as something to investigate.
	\item Add a description of the general procedure for comparing models (simplified drill string, turn off top drive controller...).
    \item Add description of fRat in Test Case chapter and add the parameter in the table.
\end{bulletedlist}

\section{Feedback}
Small differences in natural frequency might be a result of:
\begin{bulletedlist}
	\item One model is fully distributed and the other is a lumped mass model.
	\item How are the tool joints accounted for in each model?  The ExxonMobil model can handle tool joints.  Tool joints can be set to the same diameter as the drill pipe to remove this effect.  In real BHAs versus models there can be up to a 10\% difference
	\item For A-S model, $\frac{\Delta t}{\Delta x} < \textrm{speed-of-sound}$
\end{bulletedlist}

\section{Spike in Velocity}
\begin{bulletedlist}
	\item Could be caused by a change in rotational inertia.
	\item This could be tested by using a gradual change in inertia.
	\item This could be investigated by interrogating different spacial points to plot the time histories of the velocity and torque.
\end{bulletedlist}

\section{Conclusions}
To add to the conclusions:
\begin{bulletedlist}
	\item A-S model good for stick-slip and runs faster.
	\item ExxonMobil currently has axial and torsional.
\end{bulletedlist}

\section{Work to do}
\begin{bulletedlist}
	\item Convergence tests for time and discretization.
\end{bulletedlist}

\noindent Noise at bit on ExxonMobil model:
\begin{bulletedlist}
	\item Convergence tests for time and discretization.
	\item Could be numerical because there isn't damping at the bit?
	\item Possibly related to the Stribeck friction model.
\end{bulletedlist}

\noindent Other comparisons
\begin{bulletedlist}
	\item It may be possible to compare the Tulsa ``stiff string'' (second paragraph of \sectionname~\ref{sec:tulasoftstring}) model to the ExxonMobil axial degree of freedom.
\end{bulletedlist}

\section{Writing Notes}
\begin{bulletedlist}
	\item First case of open-source comparison for drill string models.
	\item Two very different models in their development and the friction models, but the results are very similar.
\end{bulletedlist}

% Modeling friction forces along the drill string is a challenging task due to the system's nonlinear behavior, continuous changes in velocity, and a large combination of materials (formation, drilling fluid, et cetera). The net frictional force is highly influenced by the sliding velocity, with the axial component contributing to drag force and the tangential component generating torque within the system.

% The axial and rotational displacement and velocity of the top drive do not instantaneously transfer to the bit. The elements of the drill string must overcome inertia and static friction to transfer energy to the next element.

% breakaway friction 