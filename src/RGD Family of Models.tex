\section{Richard-Germay-Detournay Family of Bit Model}
%\section{RGD Family of Models}
\label{ch:rgdmodels}

\subsection{Introduction}
While the current project focuses on drill string models, it is natural to couple a string model with a bit model. The Richard-Germay-Detournay family of bit models (RGD models) represents theoretical bit-rock interactions. They use a more sophisticated approach than many bit models and are, therefore, of interest. The RGD model is developed by Richard et al., 2007\ \cite{ref:richard2007a}, which simulates coupled axial and torsional vibration mode through bit-rock interaction. The model has been applied to numerous drill-string models to model bit-rock interactions.

\subsection{RGD Model}
The model combines governing equations of motions (axial and torsional) with bit-rock interface law which leads to a state-dependent delay system. In other words, it is a 2 degree-of-freedom bit model which simulates the torsional and axial degrees-of-freedom which are coupled through the bit-rock interaction model by the depth of cut. Also, the model takes into account the loss of contact at wearflat/rock interface caused by axial vibration through a discontinuous boundary condition. A brief description of the mathematical background is shown in \figurename~\ref{figure_RGD_Summary}. The parameters $T_c$, $T_f$, $W_C$, and $W_f$ in \figurename~\ref{figure_RGD_Summary} are defined with respect to the drilling regime and reflects a discontinuous boundary condition.  The drilling regimes are classified as cutting ($\omega>0, d>0$), sticking ($\omega=0, d>0$), sliding ($\omega>0, d=0$), and off-bottom ($\omega>0, d<0$), where $\omega=0$ is the angular velocity and $d$ is the depth of cut.
\begin{figure}
  \centering
  \includegraphics[width=5in]{RGD_summary}
  \caption[Mathematical description of RGD model]{Mathematical description of RGD model.}\label{figure_RGD_Summary}
\end{figure}
The base assumptions of the model are:
\begin{bulletedlist}
	\item Constant angular velocity and upward force at the top of the drill string
	\item A vertical borehole
	\item No lateral motion of the bit
	\item The energy dissipation occurs at the bit-rock interface and no additional damping parameters are used
\end{bulletedlist}

Since it is a bit model, not a drill string model, and it is limited to vertical well, this model was not selected for the evaluation. The source code of the 2 DOF RGD bit model is available.  Additional mathematical details can be found in \appendixname~\ref{ap:rgbworkflow}.

\subsection{Extended RGD Model}
Numerous drill string models benchmarked the RGD model for bit-rock interactions. \figurename~\ref{model_develop_figure} illustrates the development of the RGD model.
\begin{figure}
  \centering
  \includegraphics[width=5in]{ModelDevelop}
  \caption[RGD model development]{RGD model development.}\label{model_develop_figure}
\end{figure}
Germay extended the 2 DOF RGD model to a continuum model by applying the FEM approach in \referencename~\cite{ref:germay2009a}. This model was able to simulate the stick phase and captures higher frequency vibrations because of the increased number of degrees of freedom. Similarly, Zhang increased the DOF of the model by discretizing the drill string and implementing the spectral method. This further improved the computational efficiency by applying a Chebyshev polynomial as a basis function (see \referencename~\cite{ref:zhang2020a}). The model was able to simulate axial and torsional vibrations, including stick-slip events, with enhanced computation efficiency compared to the FEM approach. The source codes for the models were not provided, therefore, they could not be evalutated in this project. However, these models can be good candidates for further study. 