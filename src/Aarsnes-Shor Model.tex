\chapter{Aarsnes-Shor Model}
\label{ch:aarnessshormodel}

\section{Introduction}
\subsection{Overview}
The model was developed by Aarsnes and Shor (\referencename~\cite{ref:aarsnes2017a}). It is designed to simulate torsional drill string vibrations using a distributed model. It can be used both vertical and inclined wells and can incorporate BHA components. The top drive is accounted for and it contains a PI controller.

Two versions of the source code are available, one implemented in Matlab and the other in Python. The Python version is newer and is being written to improve the Matlab version. Ongoing efforts are focused on enhancing the Python version by incorporating axial motion.

\subsection{A Note on Interpretation}
In many sources, \emph{linear forces} are often represented by variables that use $f$ or $f$-like characters (f, F, $f$, $F$, \mbf{F}). In \referencename~\cite{ref:aarsnes2017a}, the variables $\mathcal{F}$, $F_s$, and $F_d$ are used denote \emph{torque} resulting from \emph{friction}.

\section{Model Description}
The model simulates torsional vibration of the drill string and includes Coulomb friction and viscous damping effects.  These are modeled as distributed along the drill string and are combined into a ``source'' term. The governing equations of motion are solved using the finite difference method (FDM) by discretizing the drill string. The assumptions of the model are:
\begin{bulletedlist}
	\item A simplified bit model (see \sectionname~\ref{sec:asbitmodel})
	\item No lateral or axial motion
    \item No lateral bending moments (soft string model)
    \item The entire drill string is in contact with the wellbore (soft string model)
	\item The effects of cuttings distribution on the friction is homogeneous
	\item Viscous damping is included
	\item Cased and open hole sections have the same friction factors
    \item Friction between the drill string and wellbore is modeled as Coulomb friction (see \sectionname~\ref{sec:asfrictionmodel})
    \item Tool joints are ignored
    \item The effect of the pressure differential, inside and outside the drill string, on the bending moment is not represented and is assumed to be negligible
\end{bulletedlist}
While tool joints are not included as part of the model, an equivalent stiffness can be manually calculated for drill string components to account for their effects.

Two different versions of the codes are available which are written in Matlab and Python. The Matlab ver.\ is more matured and has been validated with field data (\referencename~\cite{ref:aarsnes2017a}). In comparison, the Python ver.\ is a relatively recent development. Both codes are based on the same mathematic background and applies the finite difference method to solve the torsional vibration of the drill string. However, while the Python ver.\ directly solves the governing equations for angular velocity and torque separately, the Matlab ver.\ introduces Riemann invariants and combines the torque and angular velocity in the governing equations and solves at once. The comparison of the numerical implementations between the two models are explained in more detail in section \ref{AS_Math}.

\subsection{Top Drive Model}
The model contains a PI controller to represent the behavior of the top drive. The boundary conditions at the top are given by the top drive torque and velocity calculated from the top drive velocity set point. The torque on the top drive is derived as
\begin{equation}\label{AS_err}
  e = w_{sp} - w_{td}
\end{equation}

\begin{equation}\label{AS_Ie}
  I_e = \int_{0}^{t}e(\xi)d\xi
\end{equation}

\begin{equation}\label{AS_tau_m}
  \tau_{td} = K_pe + k_i I_e
\end{equation}
where $w_{sp}$ is the set point velocity, $w_{td}$ is the top drive velocity, $\tau_{td}$ is the top drive torque, and $K_p$ and $K_i$ are the proportional and integral gain, respectively. $K_p$ and $K_i$ are parameters that can be adjusted to change the response. The top drive velocity, $w_{td}$ is then obtained by dynamics
\begin{equation}\label{AS_w_td}
  \frac{\partial w_{td}}{\partial t} = \frac{1}{J_{td}}(\tau_{td} - \tau_0)
\end{equation}
where $J_{td}$ is top drive inertia, $\tau_{0}$ is the torque on top of the drill string. The angular velocity at the top of the drill string is assume to be same as top drive velocity $w_0 = w_{td}$.

\subsection{Bit Model}
\label{sec:asbitmodel}
The model is considered to be an off-bottom case.  Therefore, the bit-rock interaction is not considered.  However, an input to the model allows for a torque to be applied at the end of the drill string.

\subsection{Friction Model}
\label{sec:asfrictionmodel}
The friction is modeled as Coulomb friction with the option to set a non-zero critical velocity for the change between static and dynamic friction.  Static friction applies when the angular velocity of the drill string is below the given threshold (critical velocity). When the angular velocity exceeds the threshold, dynamic friction applies. More details can be seen from \equationname~\ref{equation_A-S_columnbfriciton}-\ref{equation_A-S_dynamic_fric}. \figurename~\ref{figure_AS_damping} illustrates the friction model for A-S model.
\begin{figure}
	\begin{minipage}[t]{\linewidth}
			\begin{minipage}[t]{0.45\linewidth}
				\centering
				\includegraphics[width=\linewidth]{AS_friction}
				\subcaption{Friction between the drill string and wellbore contact is modeled as Coulomb friction.}
				\label{fig:asfriction}
			\end{minipage}
			\hfill
			\begin{minipage}[t]{0.45\linewidth}
				\centering
				\includegraphics[width=\linewidth]{AS_friction_and_viscous}
				\subcaption{The friction model combined with viscous damping.  This is the model of the \emph{source term}.}
				\label{fig:asfrictionandviscous}
			\end{minipage}
	\end{minipage}
    \caption[Damping models for the A-S model]{Damping models for the A-S model. The friction model is shown in~(\subref{fig:asfriction}) and the combination of friction and viscous damping (source term) is shown in~(\subref{fig:asfrictionandviscous}). $F_{s}$, $F_{d}$ are the static and dynamic friction torque, respectively, $k_{T}$ is the viscous damping coefficient for torsional motion and $w_{c}$ is the critical angular velocity which is the threshold for transition between static and dynamic friction.}
	\label{figure_AS_damping}
\end{figure}
%When the angular velocity is below the critical velocity, the friction is bounded by static friction.

The Coulomb friction force $\mathcal{F}(w,x)$ is modeled as an inclusion which is shown in \equationname~\ref{equation_A-S_columnbfriciton}. The parameters for the equations can be seen in \figurename{}s~\ref{figure_AS_damping} and~\ref{figure_AS_equation_schematic}.
\begin{figure}
	\centering
	\includegraphics[width=2.5in]{AS_parameters}
    \caption[Schematic of distributed drill string in deviated well]{Schematic of distributed drill string in deviated well and indication of parameters used in the Aarnes and Shor model (\referencename~\cite{ref:aarsnes2017a}).}
	\label{figure_AS_equation_schematic}
\end{figure}
\begin{equation}
	\label{equation_A-S_columnbfriciton}
	\begin{cases}
		\mathcal{F}(w,x) = F_{d}(x), & \mbox{if } w>w_c \\
		\mathcal{F}(w,x) \in [-F_{s}(x), F_{s}(x)], & \mbox{if } |w|<w_c \ \\
		\mathcal{F}(w,x) = -F_{d}(x), & \mbox{if } w < -w_c.
	\end{cases}
\end{equation}
where $w_c$ is the critical angular velocity for the transition from static to dynamic friction, $F_{d}$ and $F_{s}$ are dynamic and static friction forces, respectively. For the condition $|w|<w_c$, $\mathcal{F}(w,x)$ is limited by the boundary values $\pm F_{s}(x)$.  For the condition $|w|>w_c$,  $\mathcal{F}(w,x)$ is equal to the dynamic friction, $F_{d}$.

The normal force, $F_N$, along the drill string is calculated according to
\begin{equation}\label{equation_AS_normal_force}
  F_n(x) = gA(x)(\rho-\rho_{mud})sin(\theta)
\end{equation}
where $A(x)$ is the cross sectional area at location $x$, and $\rho$, $\rho_{mud}$ are material and mud density, respectively.  Here $gA(x)(\rho-\rho_{mud})$ is the buoyed weight per unit length and $sin(\theta)$ accounts for the hole inclination.

The maximum torque that can be generated from the friction is from the static case and calculated as
\begin{equation}\label{equation_A-S_static_fric}
  F_{s}(x,\mu_s) = \mu_s r_o(x) F_N(x)
\end{equation}
where $F_{s}$ is the maximum torque from (static) friction, $\mu_s$ is the coefficient of static friction, and $r_o(x)$ is the outer radius of the drill string at location $x$. Similarly, the dynamic friction is calculated as
\begin{equation}\label{equation_A-S_dynamic_fric}
  F_{d}(x,\mu_s) = \mu_d r_o(x) F_N(x)
\end{equation}
where $F_d$ is dynamic frictional torque and $\mu_d$ is the dynamic friction factor.

\section{Mathematical Background}\label{SubSec_AS_mathematicalbackground}
\label{AS_Math}
\subsection{Governing Equations}
The model is based on the equations of angular motion, here named wave equation, which are
\begin{equation}
	\label{AS-motion}
	J\rho\frac{\partial w(t,x)}{\partial t} + \frac{\partial \tau (t,x)}{\partial x} = S(w,x)
\end{equation}
\begin{equation}
	\label{AS-motion1}
	\frac{\partial\tau(t,x)}{\partial t} + JG\frac{\partial w(t,x)}{\partial x} = 0
\end{equation}
where J is polar moment of inertia and $\rho$ is material density, $w(t,x)$ is angular velocity, $\tau(t,x)$ is torque and S is the source term which can be represented as
\begin{equation}
	\label{AS-sourceterm}
	S(w,x) = -k_t \rho J w(t,x) - \mathcal{F}(w,x)
\end{equation}
where $k_t$ is damping constant which is viscous shear stresses from drilling mud and cuttings bed and $\mathcal{F}(w,x)$ is Coulomb friction between the drill string and the borehole that is derived from \equationname{}s~\ref{equation_A-S_columnbfriciton}-\ref{equation_A-S_dynamic_fric}.

\subsection{Numerical Implementation of Matlab Version}
\subsubsection{Riemann Invariants}
The partial differential equations governing the motion (\equationname{}s~\ref{AS-motion}, \ref{AS-motion1}, and~\ref{AS-sourceterm}) are transformed into their Riemann invariants solved by a first order upwind scheme. The Riemann invariants are defined as
\begin{equation}\label{AS-Riemann}
  \alpha = w + \frac{c_t}{JG}\tau, \quad \beta=w-\frac{c_t}{JG}\tau
\end{equation}
where $c_t = \sqrt{G/\rho}$ is the velocity of the torsional wave. It can be noticed that angular velocity and torque can be obtained from the relationships
\begin{equation}\label{equation_Riemann_relation1}
  \frac{\alpha + \beta}{2} = w
\end{equation}
\begin{equation}\label{equation_Riemann_relation2}
  J \rho \frac{\alpha + \beta}{2c_t} = \tau
\end{equation}
and the variables $\alpha$ and $\beta$ satisfy the PDE system:
\begin{equation}\label{AS-Riemann_alpha}
  \frac{\partial \alpha}{\partial t} + c_t\frac{\partial \alpha}{\partial x} = -\mathcal{S}
\end{equation}
\begin{equation}\label{AS-Riemann_beta}
  \frac{\partial \beta}{\partial t} - c_t\frac{\partial \beta}{\partial x} = -\mathcal{S}
\end{equation}
Therefore, angular velocity $w$ and torque $\tau$ can be obtained by solving $\alpha$ and $\beta$.
The source term $\mathcal{S}$ is
\begin{equation}\label{AS-source}
  \mathcal{S} = \frac{S}{J \rho} = k_t(\alpha + \beta) + \frac{1}{J \rho} \mathcal{F}
\end{equation} \reviewcomment{Where is the ``2''/``1/2'' go?  It may reappear in \equationname~\ref{equation_A-S_columnbfriciton_Riemann}.}
where $k_t$ is damping coefficient and $\mathcal{F}$ is Coulomb friction defined in \equationname~\ref{equation_A-S_columnbfriciton}.
\subsubsection{Boundary Condition}
The boundary conditions of the drill string are given as
\begin{align}
  w_{p,top} &= w_{td} \label{AS-BC1} \\
  \tau_{c,bottom} &= \tau_{bit} \label{AS-BC2}
\end{align}
%\reviewcomment{What about $\tau_{p,top}$ and $\tau_{c,bottom}$?}   \tau_{p,top} &= \tau_{td} \\
where the subscripts $p$ and $c$ represents drill pipe and drill collar.  The input at the top of the drill string is the velocity of the top drive.  While the bit-rock interaction is not considered for this model, a torque at the end of the drill string can be applied ($\tau_{bit}$). In this project, off-bottom scenarios are used so $\tau_{bit}$ is assumed to be 0. Additionally, the conditions imposed at the drill pipe and drill collar interface are
\begin{align}
  w_{p,interface} &= w_{c,interface} \label{AS-interface1} \\
  \tau_{p,interface} &= \tau_{c,interface} \label{AS-interface2}
\end{align}
where $interface$ is used to denote the connection of string elements with different properties ($c_t$, $J$, or $G$). The boundary and interface condition is then expressed for $\alpha$ and $\beta$. \figurename~\ref{AS_discretizeDS} illustrates the schematic view of the discretized drill string with boundary conditions and interface conditions in terms of Riemann invariants.

\equationname{}s~\ref{AS-BC1} and~\ref{AS-BC2} can be re-written from the relationships in \equationname{}s~\ref{equation_Riemann_relation1} and \ref{equation_Riemann_relation2} as
\begin{align}
  \alpha_{p,top} &= -\beta_{p,top} + 2w_{te} \label{AS-riemannBC1} \\
  \beta_{c,bottom} &= \alpha_{c,bottom} - 2\tau_{bit} \frac{c_t}{J_c G_c} \label{AS-riemannBC2}
\end{align}
and \equationname{}s~\ref{AS-interface1} and~\ref{AS-interface2} can be re-written as
\begin{align}\label{AS-riemanninterface}
	\beta_{p,bottom} &= \frac{1}{1+\overline{Z}}\left(\alpha_{p,bottom}(1-\overline{Z}) + 2\overline{Z}\beta_{c,top} \right) \\
	\alpha_{c,top} &= \frac{1}{1+\overline{Z}}\left(2\alpha_{p,bottom} - \beta_{c,top}(1-\overline{Z})\right)
\end{align}
where $\overline{Z}$ is the relative magnitude of the impedance calculated as shown below.
\begin{equation}\label{AS_Zbar}
	\overline{Z} = \dfrac{\left(\dfrac{c_t}{JG}\right)_{p,bottom}}{\left(\dfrac{c_t}{JG}\right)_{c,top}}
\end{equation}

\begin{figure}
	\centering
	\includegraphics[width=6in]{AS_discretizedDS}
	\caption[Schematic of discretized drill string and boundary conditions]{Schematic of discretized drill string, boundary conditions, and interface conditions based on Riemann invariants $\alpha$ and $\beta$.}
	\label{AS_discretizeDS}
\end{figure}

\subsubsection{Forward Scheme}
The PDE is solved using an upwind scheme. First, the Coulomb friction is updated by,
\begin{equation}
    \label{AS-upwind}
    \mathcal{F}_{j}^k = \frac{\rho J}{2 \Delta t}\left(\alpha_j^k - c_t \frac{\Delta t}{\Delta x}(\alpha_j^k - \alpha_{j-1}^k) + \beta_j^k + c_t \frac{\Delta t}{\Delta x}(\beta_{j+1}^k-\beta_j^k) + 2{\Delta t} k_t (\alpha_j^k + \beta_j^k)\right)
\end{equation}
where $\Delta x$ is the cell size, $\Delta t$ is the time step size, $j$ is the cell number, and $k$ time step.\footnote{\equationname~\ref{AS-upwind} is slightly different than that presented as \equationname{}s~22 and~23 in \referencename~\cite{ref:aarsnes2017a}.  There is believed to be a typo in the reference.}  In the Matlab ver.\ of the code, \equationname~\ref{AS-upwind} is arranged slightly differently and presented in \equationname~\ref{AS-upwind-matlab}.\footnote{\equationname~\ref{AS-upwind-matlab} differs from the code in the last term by the absence of the $\Delta t$ variable.  It is believed that the line of code \textcode{\TINY Pvf = dt*p.kt*(Pbeta+Palpha)} should not contain the \textcode{\TINY dt} variable.}
\begin{equation}
    \label{AS-upwind-matlab}
    \mathcal{F}_{j}^k = \frac{\rho J}{2}\left(\frac{\alpha_j^k}{\Delta t} - \frac{c_t}{\Delta x}(\alpha_j^k - \alpha_{j-1}^k) + \frac{\beta_j^k}{\Delta t} + \frac{c_t}{\Delta x}(\beta_{j+1}^k-\beta_j^k) + 2 k_t (\alpha_j^k + \beta_j^k)\right)
\end{equation}

\equationname~\ref{AS-upwind} is driven by rearranging \equationname{}s~\ref{AS-Riemann_alpha}-\ref{AS-source}. Then, $\mathcal{F}_{j}^k$ is limited by \equationname~\ref{equation_A-S_columnbfriciton} and \ref{equation_A-S_dynamic_fric}, which can be expressed in Riemann invariants as
\begin{equation}
	\label{equation_A-S_columnbfriciton_Riemann}
	\mathcal{F}_{j}^k =
	\begin{cases}
		max(min(\mathcal{F}_{j}^k,{F_C}_j^k),-{F_C}_j^k) , & \mbox{if } \frac{|\alpha_j^k + \beta_j^k|}{2} \le w_c \\
		sign\left(\frac{|\alpha_j^k + \beta_j^k|}{2}\right) {F_C}_j^k f_{rat} , & \mbox{if } \frac{|\alpha_j^k + \beta_j^k|}{2} \ge w_c \\
	\end{cases}
\end{equation}
where ${F_C}_j^k$ is the static friction calculated at cell $j$ and time step $k$.\footnote{\equationname~\ref{equation_A-S_columnbfriciton_Riemann} follows the code in the Matlab ver.\ and appears different than \equationname~26 from \referencename~\cite{ref:aarsnes2017a}. The code version and the paper version of the equation may not be equivalent; it is believed the code version is correct.  Further investigation is suggested to sort out the differences.} Finally, the Riemann invariants are updated iteratively by using an upwind scheme using the following equations.\footnote{\equationname{}s~\ref{equation_upwind_alpha} and~\ref{equation_upwind_beta} are slightly different than \equationname{}s~25 and~26 from \referencename~\cite{ref:aarsnes2017a}. It is believed there is a typo and the $\frac{1}{J \rho}$ term was left out.}$^,$\footnote{\equationname{}s~\ref{equation_upwind_alpha} and~\ref{equation_upwind_beta} are slightly different than the implementation in the Matlab ver.\ of the code. It is believed that the line of code \textcode{\TINY Pvf = dt*p.kt*(Pbeta+Palpha)} should not contain the \textcode{\TINY dt} variable which creates the differences between the code and the equations presented here.}
\begin{equation}
	\label{equation_upwind_alpha}
	\alpha_j^{k+1} = \alpha_j^{k} - c_t \frac{\Delta t}{\Delta x}(\alpha_j^k - \alpha_{j-1}^k) - \Delta t \left( k_t (\alpha_j^k + \beta_j^k) + \frac{1}{J \rho} \mathcal{F}_j^k \right)
\end{equation}
\begin{equation}
	\label{equation_upwind_beta}
	\beta_j^{k+1} = \beta_j^{k} - c_t \frac{\Delta t}{\Delta x}(\beta_{j+1}^k - \beta_{j}^k) - \Delta t \left( k_t (\alpha_j^k + \beta_j^k) + \frac{1}{J \rho} \mathcal{F}_j^k \right)
\end{equation}

\subsection{Numerical Implementation of Python Version}
The Python ver.\ directly solves \equationname{}s~\ref{AS-motion} and \ref{AS-motion1} by a forward-time-centered-space FDM rather than introducing Riemann invariants like in the Matlab ver. The angular velocity and torque are updated iteratively from the algorithm below. The parameters in the algorithm are listed in \tablename~\ref{AS_inptut_params}.
\begin{table}
	\centering
	\begin{tabularx}{\linewidth-0.75in}{|c|c|L|}
	\hline
	\tablecolumnheadervlinesone{Parameter} & \tablecolumnheadervlinestwo{Units} & \tablecolumnheadervlinestwo{Notes} \\
	\hline
	Simulation length & $s$ & Total simulation time \\
	\hline
	$w_c$ & $rad/s$ & Cut-off angular velocity (static, dynamic transition) \\
	\hline
	$\mu_s$ & -& Static friction factor \\
	\hline
	$\mu_k$ & - & Kinetic friction factor \\
	\hline
	$K_t$ &- & Inertial torsional damping \\
	\hline
	$K_s$ &- & Viscous damping coefficient \\
	\hline
	$\rho$ & $lbf/ft_3$ & Drill string density \\
	\hline
	G & $lbf/ft^2$ & Shear modulus \\
	\hline
	$K_p$ & $lb$-$ft$ & P-gain of top drive (PI controller) \\
	\hline
	$K_i$ & $lb$-$ft/s$ &I-gain of top drive (PI controller) \\
	\hline
	$J_{TD}$ & $lb$-$ft^2$ & Top drive inertia \\
	\hline
	Ramp speed & $RPM/s$ & Ramp speed of top drive \\
	\hline
\end{tabularx}
\caption[Input parameters of Aarsnes-Shor model (Python ver.)]{Input parameters of Aarsnes-Shor model (Python ver.). Well trajectory, top drive set velocity, and bit constant are the additional parameters which are not included in this table.}\label{AS_inptut_params}
%\end{tabular}
\end{table}

\newcommand{\codecomment}[1]{\hfill #1}
\pushinitialcodeindent{0in}
\begin{code}[\codenumbering]{}
\codeitemnonumber \pseudocodefor{} i = 1 simulation length/$\Delta t$
	\stepcodelevel{}
	\codeitemnonumber nstep = freq/$\Delta t$ \codecomment{Freq=frequency of removing the noise}
	\codeitemnonumber \pseudocodefor{} {j=1:nstep}
		\stepcodelevel{}
	    \codeitemnonumber 1. Set the top drive set point torque and angular velocity
	    \codeitemnonumber $w_{sp}(t+1)=w_{sp}(t)\pm ramp \cdot dt$
	    \codeitemnonumber $e=w_{sp}(t+1)-w_{td}(t), \quad I=e \cdot dt$
	    \codeitemnonumber $\tau_{td}(t+1) \gets \tau_{td}(t) + (k_e e + k_i I) \cdot dt$
	    \codeitemnonumber $w_{td}(t+1) \gets w_{td}(t) + (\tau_{td}(t+1)-\tau_{top}(t))/J_{td} \cdot dt$
	    \codeitemnonumber 2. Bit rotation
	    \codeitemnonumber $\tau_{bit}(t+1) \gets bitconstant \cdot w_{bottom}(t)$
	    \codeitemnonumber $w_{bit}(t+1) \gets w_{bottom}(t)$
	    \codeitemnonumber 3. Update top drive torque
	    \codeitemnonumber $\tau_{TD}(t+1) \gets \tau_{TD}(t+1)-J_{TD}(w_{TD}(t+1)-w_{TD}(t))/dt$
	    \codeitemnonumber 4. Calculate source term
	    \codeitemnonumber $fric \gets \mu_k \cot f_n \cdot r_o$ \codecomment{Coulomb friction}
	    \codeitemnonumber $Iner \gets k_t \cdot \rho \cdot J \cdot (w(t)-w(t-1))$ \codecomment{Inertia}
	    \codeitemnonumber $vis \gets k_s \cdot w(t)$ \codecomment{Viscous damping}
	    \codeitemnonumber $S \gets (fric+iner+vis)/dx$
	    \codeitemnonumber 5. Update the torque and angular velocity \codecomment{eqn.\ref{AS-motion} and \ref{AS-motion1}}
	    \codeitemnonumber $\tau(t+1) \gets \tau(t) - JG \cdot dl/dt (w_n(t)-w_{(n-1)}(t))$ \codecomment{n is discretized number}
	    \codeitemnonumber $w(t+1) \gets w(t) 2 \cdot dt/(J \cdot \rho)\left[(\tau_n(t)-\tau_{n-2}(t))/(2*dx)+S\right]$
	    \prevcodelevel{}
	\codeitemnonumber \pseudocodedonefor{}
	\codeitemnonumber $\overline{w} \gets \overline{w} \cdot Kernel$ \codecomment{remove high frequency noise}
	\codeitemnonumber $\overline{\tau} \gets \overline{\tau} \cdot Kernel$
	\prevcodelevel{}
\codeitemnonumber \pseudocodedonefor{}
\end{code}
\popinitialcodeindent{}

\section{Comparison of the Matlab and Python Versions}
The Matlab and the Python versions were compared and some findings are reported in this section.  The tests were conducted based on Test Case 1, please refer to \chaptername~\ref{ch:testcases} for a detailed description and the input parameters.

\subsection{Effect of PI Controller}
The A-S model contains a PI controller on top drive for a more realistic simulation. However, the results were sensitive to the top drive related input parameters $k_p$, $k_i$, and $J_{TD}$ and to the ramp speed. \figurename~\ref{figure_topdrive_sensitivity} illustrates the comparison between Matlab ver.\ and Python ver.\ when same top drive related parameters are used. Please note that the Matlab ver.\ uses metric units and the Python ver.\ uses imperial units. See \tablename~\ref{table_topdrivesensitivity_input} for the parameter values used as input.

\begin{table}
    \centering
	\begin{testcasetable}
		$k_p$ & $38e^3 \; lbf\cdot ft/s $ & $28e^3\; Nm/s$ & P-gain \\
		\hline
		$k_i$ & $100e^3 \; lbf\cdot ft$ & $74e^3\; Nm$  & I-gain \\
		\hline
		$J_{TD}$ & $2900 \; lb\cdot ft^2 $ & $ 68818 \; kgm^2$ & Top drive inertia \\
		\hline
	\end{testcasetable}
	\caption[Top drive related parameters for comparison]{Top drive related parameters for comparison between A-S model Matlab and Python versions.}
	\label{table_topdrivesensitivity_input}
\end{table}

\begin{figure}
	\centering
	\includegraphics[width=6.5in]{PI_comp}
    \caption[Comparison of drill string response to same top drive parameters]{Comparison of top drive and bit responses with same top drive parameters between the Matlab ver.\ and the Python ver. The first and second columns are the results from the Matlab ver.\ and the Python ver., respectively}
	\label{figure_topdrive_sensitivity}
\end{figure}
The results imply that, for the further study, the exact input parameters are required or the PI controller should be removed. Therefore, in the codes the top drive was removed from the system and the velocity at the top of the drill string was assumed. This simplified the comparison between the different models. The flowchart in \figurename~\ref{figure_Topdriveremove_math} shows how the algorithms were modified to remove the PI controller.

\begin{figure}
	\centering
	\includegraphics[width=6in]{Topdriveremove_math}
    \caption[Flow chart of top drive elimination]{Flow chart of top drive elimination. The modification of the model assumes only the drill string in the model and it rotates with given velocity at the top of the drill string.}
	\label{figure_Topdriveremove_math}
\end{figure}

A comparison of the models with and without a PI controller is shown in \figurename{}s~\ref{figure_topdriveremove_Matlab} and \ref{figure_topdriveremove} for the Matlab and Python versions, respectively. The angular velocity and torque for both the top drive and bit are depicted. In the modified codes (no PI controller), the top drive velocity and torque will refer to those at the top of the drill string. The original and modified codes showed similar behavior except during the first stage when the top drive velocity increased from 0 to 40 $RPM$ (from 0 to 1 second). At the end of this initial ramp up, spikes (high magnitude peaks) can be seen in the top drive torque when the PI controller is included (see \figurename{}s~\ref{figure_topdriveremove_Matlab} and \ref{figure_topdriveremove}). Additionally, decaying of the vibration amplitude was observed when the simulation was run with the PI controller, which can be clearly seen in the Matlab ver. This is due to the damping effect from the PI controller managing the top drive velocity. Eventually, vibrations are removed and the bit velocity converges to the top drive velocity.

In summary, removing the PI controller enables us to obtain a model with reduced parameters and eliminates the damping effect from the PI controller. This simplification of the model results in an advantage in model comparison. However, if the PI controller is available in the rig, simulation with PI controller can provide more accurate results when comparing the model to a field application.
\begin{figure}
  \centering
  \includegraphics[width=6.5in]{PI_matlab}
  \caption[Comparison with and without a PI controller for the Matlab ver.\ of the A-S model]{Comparison with and without a PI controller for the Matlab ver.\ of the A-S model. The angular velocity and torque of the top drive and bit are shown. The first and second columns represent the results with and without PI controller, respectively.}
  \label{figure_topdriveremove_Matlab}
\end{figure}

\begin{figure}
  \centering
  \includegraphics[width=6.5in]{PI_python}
  \caption[Comparison with and without a PI controller for the Python ver.\ of the A-S model]{Comparison with and without a PI controller for the Python ver.\ of the A-S model. The angular velocity and torque of the top drive and bit are shown. The first and second columns represent the results with and without PI controller, respectively. Removing the PI controller removed the spikes (high magnitude peaks) at the end of this initial ramp up.}
  \label{figure_topdriveremove}
\end{figure}

\subsection{Comparison Between the Matlab and Python Versions}
A comparison between the two versions was conducted using the scenario from Test Case 1. The Python ver.\ showed different patterns of bit and top drive behavior compared to the Matlab ver. There was a period where the bit angular velocity and the top drive torque were maintained at the maximum value for about 2 seconds. There was a similar length ``stuck'' phase. The comparison between the Matlab ver.\ and Python ver.\ can be seen in \figurename~\ref{figure_Test1_comp_chASmodel} with results from the ExxonMobil model added for reference.

It was suspected that the different results in the Python ver.\ were caused by the difference in wave travel velocity along the drill string. This is proportional to $\sqrt{\rho/J}$.  Therefore, a test was done and the density was decreased to $1/4$ of the default input (490.50 $lb/ft^3$ $\rightarrow$ 122.65 $lb/ft^3$). The results improved somewhat and were closer to the Matlab ver.\ and the ExxonMobil model (see \figurename~\ref{figure_Python_reducedDensity}). However, there is still a small stuck phase and phase of maximum velocity.

Given the limited amount of time available, it was not possible to track down this discrepancy. Therefore, only the Matlab ver.\ of the A-S model will be used for the final results and discussions in \chaptername~\ref{ch:results}.

\emph{Models of this nature are complex physical simulations and require significant training and study to operate. In addition, the Python ver.\ has been previously verified against the Matlab ver. Together, these suggest that continued study of the model will unearth the source of the discrepancy as a minor misinterpretation of input or feature. This should not be interpreted as an error in the model; the source of the discrepancy is currently unknown. It is left as future work to find the cause of the issue.}

\begin{figure}
  \centering
  \includegraphics[height=4.5in]{version_comp}
  \caption[Comparison between different models for Test Case 1]{Comparison of the angular velocity of bit and the top drive between the A-S model Matlab ver., A-S model Python ver., and ExxonMobil model for Test Case 1. The Matlab ver.\ and the ExxonMobil model match well. The Python ver.\ exhibited a period where the bit angular velocity reached a maximum and maintained it for about 2 seconds. It also exhibited a stick phase for approximately the same length of time. This should not be interpreted as an error in the model; the source of the discrepancy is currently unknown and finding it is left as future work.}
  \label{figure_Test1_comp_chASmodel}
\end{figure}

\begin{figure}
  \centering
  \includegraphics[width=6.5in]{density_comp}
  \caption[Effect of drill pipe density in Python ver.\ for Test Case 1]{Effect of drill pipe density to drill string vibration for A-S model, Python ver.\ for Test Case 1. The first and second columns show the results with $\rho$ = 409.5 $lb/ft^3$ and  $\rho$ = 122.65 $lb/ft^3$, respectively.}\label{figure_Python_reducedDensity}
\end{figure}


