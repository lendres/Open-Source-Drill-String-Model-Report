\chapter*{Executive Summary}
\addcontentsline{toc}{chapter}{Executive Summary}

The purpose the project \emph{Research on Numerical Simulation of Downhole Drill String Dynamics} was to establish background knowledge related to drill string models.  Key objectives of the project were to review existing models, establish a set of test cases, document procedures for comparing models, and evaluate available models.

Several open-source and published models were reviewed to assess their features.  Models that did not pertain directly to drill strings, lacked key features, or the source code was unavailable were then filtered out and the remaining models selected for evaluation.  A set of Test Cases was then established and the models chosen for evaluation run on those test cases in order to provide a direct comparison.

% A set of procedures and methods for evaluating drill string models was established and documented.

The first model selected was the model from Aarsnes and Shor. Two versions of the model are available, a more mature version in Matlab and an upgraded version written in Python that is still in development.  It is a distributed model in that the governing equations of torsional motion are solved by finite difference method. The model assumes constant torque on bit during drilling. Friction is modeled by the Coulomb friction and viscous damping. Coulomb friction is modeled as a jump, where the dynamic friction is calculated as a certain ratio of static friction. Also transition from static to dynamic friction occurs when the torsional velocity overcomes the critical velocity. 

The second model selected was developed at ExxonMobil and is written in Python. The model is a coupled axial-torsional system.  This allows for thorough examinations into the system's behavior under diverse operational settings. The precise representation of axial and tangential forces, as well as the explanation of stick-slip reasons, is achieved in the model through the integration of friction models and the bit-rock interaction model. The system incorporates two degrees of freedom per node, encompassing both axial and rotational displacements as well as velocities. The governing equations are solved using the ODE45 solver and subsequently verified by comparison with field data. Significantly, the model considers the trajectory of the well, the effects of buoyancy, and uniform mud drag in order to accurately calculate the friction force. Its versatility extends to simulating Mud-motor and heave compensator dynamics. Additionally, the model's Python-based source code is designed with modularity, ensuring ease of use and offering comprehensive guidance for effective utilization. 

The simulation results from both models for all test cases were compared and evaluated. The results from both model matched well showing similar behavior of angular velocity and torque on top drive and bit. Specifically both models showed similar amplitude and frequency of the drill string vibration and a similar response during stick-slip.

The availability of these well-structured Test Cases allows for rigorous comparison and verification of drill string models, enabling researchers and engineers to assess their model's accuracy and performance in various wellbore scenarios. The data and findings obtained from these Test Cases offer valuable insights and constitute a substantial contribution to the field of drill string modeling and simulation. As a result, this study serves as a valuable reference for further research and development in the domain of drilling engineering and related disciplines. 
