\chapter{Introduction}
This document was created as part of the project \emph{Research on Numerical Simulation of Downhole Drill String Dynamics}, a fundamentals research project at CNPC USA.  The project was proposed and kicked off in 2023.

The concept for a model review and evaluation result from conversations with members of the \osdc{} (\referencename~\cite{ref:pastusek2019a}).  A number of models have been released to the community's \emph{Github} site, however, it was not known exactly how the models compare in terms of features and performance.  At the same time, CNPC was both interested in studying drill string models and is a member of two industry-university research programs.  These programs may provide access to drilling dynamic models.  These models also need to be reviewed to determine features and investigated to determine availability.

Based on this it was decided to review the drilling dynamic models on the \osdc{} as well as others that may be accessible through other industry memberships.  Promising drill string models would then be selected for further evaluation and testing.

\section{Outline}
The models associated with drilling dynamics that were available on the \osdc{} Github site, as well as those potentially available through industry-university research programs were reviewed.  However, not all of these models were exercised.  Some of the models did not pertain directly to drill strings, lacked key features, or the source code was unavailable.

Two models that passed all the criteria were selected for in-depth evaluation.  The first is the torsional model of Aarsnes-Shor.  The model has two versions available; the more mature Matlab version and a newer port to Python.  The second model is from ExxonMobil and written in Python.  The theory and source code of both models was examined.

To compare the models, a set of Test Cases was developed.  The procedures, methods, and reasoning for selecting the Test Cases is documented.  The models were then run on all the Test Cases and the results compared.  A final set of conclusions and recommendations are reported.

\section{Objectives}
The primary motivation for CNPC USA was to evaluate soft string models.  However, in order to compare models, it is necessary to have a set of Test Cases to use as a benchmark.  In addition, well defined procedures should be used in order to allow a direction comparison.  Based on these requirements, the following were adopted as objectives.
\begin{numberedlist}
	\item \emph{Review} all the models that were potential available to determine their features.
	\item \emph{Establish} a set Test Cases for comparing drill string models.
	\item \emph{Document} procedures and methods used in the comparisons.
	\item \emph{Evaluate} soft string models and assess their features and performance.
\end{numberedlist}

\section{Organization of this Document}
This document is organized in order to facilitate different use cases.  Together, the chapters provide a complete description of a process that starts from a model search and ends with a comparison between models.  In addition, some of the chapters are written as stand alone chapters that can be referenced independently.  Other chapters are naturally related to each other and are best read as a unit.

\chaptername~\ref{ch:modelreview} can be read as a stand alone chapter.  It contains short sections that describe each of the models that were examined.  While the models were reviewed in the context of the current project, the reviews can serve as a reference of available drilling dynamics models.

\chaptername~\ref{ch:testcases} can be read as a stand alone chapter.  The Test Cases are generalized and not specific to the particular models evaluated.  The intent is provide baseline Test Cases that can be used to compare any drill string models.  The Test Cases and procedures are defined completely within the chapter.

In \chaptername{}s~\ref{ch:aarnessshormodel},~\ref{ch:exxonmobilmodel}, and~\ref{ch:results} two different models are described in detail and compared on the Test Cases.  These chapters are best read as a unit.

\section{Important Notes on the Models}
It should be noted that these models are considered research models and are currently still in development.  As such, it was expected to lack some features a commercial code would have.  In addition, some issues were expected to be encountered.  Any issues are reported solely in the interest of helping to improve the models and providing important feedback to the developers.  \important{At no time should any comments in this report be construed as criticisms of these models.}

On the contrary, this collaboration is viewed as demonstrating the benefits of open-source software.  A wider audience gets to make use of the software and access to the source code means issues can examined, tested, and feedback provided to the developers.

%The team is eternally grateful for allowing access to these models.
\section{Nomenclature}
\subsection{Steady-State and Transient}
In vibration analysis, the terms \emph{transient} and \emph{steady-state} refer to the way in which the vibration of a system behaves over time.  Transient vibration is a vibration that changes over time, while steady-state vibration is a vibration that remains constant (or consistent) over time.

In classical mechanics and dynamics, these are well defined and understood.  Steady-state vibrations refer to harmonic vibration that represents the particular part of the solution to the governing equations of motion.  The transient response is assumed to be damped out so that it has diminished to zero.

A discrete system with repeating, but non-harmonic loading, would not strictly fall into this definition.  It differs from the classical system in the steady-state phase.  The classical steady-state displays an exactly repeating pattern of behavior.  The discrete, non-harmonic systems has a phase where the initial, non-repeating load has been damped out.  However, the remaining ``steady-state'' may exhibit slight differences from cycle to cycle of vibration.

However, there are obvious similarities between the classical and discrete, non-harmonic loading cases.  A clear response to machine start up and other changes in loading is evident.  Similarly, it can be seen that, given time, the systems will damp out their response to these changes.  So while the classical and the discrete, non-harmonically loaded systems exhibit somewhat different traits, they are comparable in the broad sense.  As such, it is useful to speak of the two systems in the a similar manner.  \emph{Therefore, in direct analogy to classical harmonic vibration, we will refer to the transient and steady-state phases of the discrete, non-harmonic system.}

\subsection{Dry, Viscous and Total Friction}
The governing equation of the motion can be represented as 
\begin{equation}\label{eq:motion_GE}
    m\ddot{x} + c\dot{x} + k\dot{x} + f_{fric} = F_{ext}
\end{equation}
where $m$ is mass (mass moment of inertial), $c$ is damping coefficient, $k$ is axial stiffness (torsional stiffness), $f_{fric}$ is frictional force, $F_{ext}$ is external force (external torque) and x is axial displacement (angular displacement) for axial (torsional) mode. In drilling dynamics, the damping force, $c\dot{x}$, is a drag, a type of friction for drill string motion from drilling mud and cuttings. Therefore, the damping force is also called viscous friction in this project. On the other hand, frictional force, $f_{fric}$, is dry friction which is a friction between drill string and wellbore surface, also opposing the motion of the drill string. Therefore, total friction in the system is defined as a combination of viscous friction and dry friction.

\subsection{Models}
\subsubsection{Static and Dynamic}
Models can be classified as either static or dynamic.  Static models do not account for acceleration (\staticforcebalance{}).  Dynamic models account for inertia (\dynamicforcebalance{}).  Within each of these categories, there are a variety of types of models (single body/multibody, rigid body/flexible, et cetera).  Particularly for dynamics models, there is a broad range of fidelities (or complexity).  The authors did not feel the compulsion to name or classify each of these types of dynamics models.  Rather, a simpler, more straightforward approach is taken and the features of each model are described.

\subsubsection{Steady-State and Transient}
``Steady-state model'' is a synonym for ``static model.'' The phrase ``transient model'' is sometimes used to mean a dynamic model that can capture transient phases.  The authors prefer the terms ``static'' and ``dynamic'' models.  Partially because they are more approachable and easier to understand.  But also as a way to prevent confusion.  The terms ``steady-state'' and ``transient'' will be reserved for the steady-state and transient phases of a dynamic response.

\subsection{Review and Evaluation}
In this document we will use the term ``review'' in context of a model to mean a literature review and assessment of the features of the model.  The term ``evaluation'' will be used to mean the model was executed, run, tested on standardized cases, and compared to other models.

\section{Guidance and Technical Committee}
Several members of the industry and academia played key roles in this project.  Their support and guidance is greatly appreciated.  Without their assistance, this project would not have been possible.  Brief biographies are presented below.

\begin{committeemember}{Rajat Dixit}{MS from Indian Institute of Technology}{Wells Engineer (Drilling) at ExxonMobil}
Expertise includes fluids and thermal sciences, FEA, computational sciences, solid mechanics and heat transfer.
\end{committeemember}

\begin{committeemember}{Eduardo Gildin}{Ph.D. from University of Texas}{Professor at Texas A\&M}
Research interests include model reduction of large-scale dynamical systems, control and optimization of large-scale dynamical systems and reservoir modeling and simulation
\end{committeemember}

\begin{committeemember}{Anirban Manna}{MS from Indian Institute of Technology}{Wells Research Engineer at ExxonMobil}
Expertise includes FEA, fatigue, buckling, data analysis and drilling dynamics.
\end{committeemember}

\begin{committeemember}{Paul Pastusek}{MBA from University of Houston}{Drilling Mechanics Advisor at ExxonMobil}
Areas of interest include rig automation, debottlenecking, forensics analysis, drilling dynamics, borehole quality and technical training
\end{committeemember}

\begin{committeemember}{Greg Payette}{Ph.D. from Texas A\&M}{Wells Research Engineer at ExxonMobil}
Research interests include drilling dynamics, wellbore quality, open-source software, FEA and drilling automation
\end{committeemember}

\begin{committeemember}{Roman Shor}{Ph.D. from University of Texas}{Associate Professor at University of Calgary}
Research interests include in the areas of drill string dynamics modelling and control, drilling optimization and drilling systems automation.
\end{committeemember}





