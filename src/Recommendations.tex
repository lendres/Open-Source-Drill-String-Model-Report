\chapter{Conclusions and Recommendations}
\section{Conclusions}
This project served as a test case for open-source software.

This project also served as a test case for collaborative research.  The binding of the individuals to the project was not contracts and payments.  It was a mutual interest in advancing the art.

Considering the overall excellent comparison of the model, it may be asked if a deep investigation into the friction models in warranted. The authors feel that it is for a few reasons.  First, a complete knowledge of the models and their limitations is required to be able to apply and interpret them in respect to real world physics.  Second, the friction factor is a collection of unknowns.  By which it is meant, any information that is incomplete or not modeled is captured by this term and then the term is adjusted until it matches reality.  Any narrowing of this window of unknown provides opportunities for a greater understanding of the physics and improved predictions of the models.  Third, the friction models may not have a large impact on events happening on longer timescales, but high frequency events may be effected by these choices.  Research must be conducted to determine the answer.

\begin{bulletedlist}
	\item Both Stribeck and Coulomb friction models can reproduce stick slip dynamics.
	\item Stribeck and Coulomb friction can produce similar results.
	\item The Stribeck friction is slightly easier to implement algorithmically as the friction value can be directly calculated without having to check if static or dynamic friction applies.
\end{bulletedlist}

\section{Next Steps}
\subsection{Remaining Work}
\begin{bulletedlist}
	\item Plot bit velocity and top drive torque for Test Case 2b and 4b.
	\item Plot critical velocity of friction model for A-S model at 5 and 10 $RPM$.
	\item Plot A-S model with Stribeck friction.
	\item Develop procedures for comparing models with a top drive.
	\item Test viscous damping in both A-S and ExxonMobil models.  The damping coefficients may not be in the same units and this needs to be checked.
	\item Explore \emph{The Case of the Missing Two}\footnote{\equationname~\ref{AS-source} seems to be missing the ``2'' from \equationname~\ref{equation_Riemann_relation1}}.
	\item Investigate why the friction (source term) seems out of bounds of the static window and dynamic line on the friction plot.
	\item Continuation of friction effect study (Coulomb vs Stribeck, outliers, et cetera)
	\item Evaluation of viscous damping
	\item Implicit versus explicit solvers
	\item Resources (what do we have, what can we add)
	\item Collaboration on code (use Github to contribute pull requests, et cetera)
	\item Mud motor effects
	\item Cased versus open-hole feature addition and case study
	\item Investigate the selection of critical velocities when comparing Coulomb and Stribeck friction models
\end{bulletedlist}

\subsection{Test Cases}
The Test Cases selected were deliberately kept simple to provide the foundation of model comparisons.  More advanced Test Cases are necessary.  It should be noted that the more advanced the Test Case, the more difficult the comparisons will be.

Some example Test Cases that could be performed next include the following.
\begin{bulletedlist}
	\item Increase/decrease the moment of inertia to observe the change in frequency, the expected results are easily calculated
	\item Develop on-bottom test cases
\end{bulletedlist}

\section{Path Forward}
\subsection{Collaboration of Industry and Academia}
Key collaborations both within the industry and between industry an academia can be highly beneficial.  Development of a commercial grade drill string software is a complex and expensive operation.  Collaboration can both lower overall costs and accelerate projects.

This project combined personnel from multiple industry and academia entities.  Even in this relatively small scale project, the combined knowledge and resources of all these entities greatly resulted in an observable decrease in time and increase in productivity.

\subsubsection{Commercialization Collaboration}
It can be observed on the academic side that there are many knowledgable and skill individuals in developing the theory and mathematics behind drill string codes.  What academia often lacks is the resources to run a large scale software development project.  While industry has skill individuals capable of understanding the theory and math required to develop a drill string code, that learning can take time.  At the same time, industry can often contribute development resources that a university may have a hard time justifying.  It seems natural, therefore, that a partnership between academia and industry could leverage advantages from both sides.

\subsection{Development of Generalized User Interface}
All drill string models have a common need for certain input.  These items include the wellbore geometry, drill string definition, fluid properties, et cetera.  The drill string can further be broken down into the length, inside diameter, outside diameter, material, et cetera of each tubular section.  An important component of user friendly drill string software application is a graphical user interface to quickly and easily input this information.  While the exact format of this information may change from model to model, the content remains the same.

Therefore, it seems feasible to create a universal ``front end'' for a drill string code that could serve as a user friendly interface to any model.  In such a design, there would likely require an intermediate ``interpreter'' to convert the information to the require format.  However, this \emph{Interpreter} could be a lightweight object that simply marshes data into the necessary format for a particular model.

\subsubsection{Requirements}
To develop a universal front end, a few requirements would have to be met.  Use of an interpreter interface would seem a natural way to convert to each models particular formats.  An alternate approach would be to require each model to conform to a particular input.  The front end would need to allow for extensions to the software to meet particular input required by a specific model but not used in others.  This could be in the form of an ``add on'' or in the case of an open-source front end these could be added directly.  Care would have to be taken to separate out these sections of code so it is clear they are specific and not universal.

\subsubsection{Potential Pitfalls}
While is seems reasonable that a universal interface could be developed, it would need to be carefully evaluated.  Trying to create a piece of software that is too general often leads to an over complex piece of code that is difficult to understand. 